\documentclass[%
pdf,
%nocolorBG,
colorBG,
slideColor,
%slideBW,
%draft,
%frames
azure
%contemporain
%nuancegris
%troispoints
%lignesbleues
%darkblue
%alienglow
%autumn
]{prosper}
\usepackage{amsmath}
\begin{document}

\begin{slide}{Overview of Programs}

Dept. of Computer Science

University of California, Davis

http://www.cs.ucdavis.edu

\bigskip

Professor Norm Matloff 

matloff@cs.ucdavis.edu

\bigskip

Hard copy of presentation:
http://heather.cs.ucdavis.edu/ucdcs.pdf 

\end{slide}

\overlays{3}{

\begin{slide}{Computer-Related Majors}

\begin{itemstep}

\item Computer Science (CS):  

Administered by CS Dept., degree in Letters and Science

\item Computer Science and Engineering (CSE):  

Administered by CS Dept., degree in Engineering

\item Computer Engineering (CE):  

Administered by ECE Dept., degree in Engineering

\end{itemstep}

\end{slide}

}

\begin{slide}{Curricular Comparison}


{\centering \begin{tabular}{|c|c|c|c|c|}

\hline
major & college & software & hardware & freedom \\ \hline
\hline
CS & L\&S & heavy & minimal & lots \\ \hline
CSE & Eng. & heavy & strong & $\approx 0$ \\ \hline
CE & Eng. & substantial & heavy & $\approx 0$ \\ \hline

\end{tabular}\par}

\bigskip

All of these majors have an integrated BS/MS option.

\end{slide}

\begin{slide}{Which Is Better, CS or CSE?}

In terms of jobs prospects for new grads, there is essentially no
difference.

\end{slide}


\overlays{3}{

\begin{slide}{Then What \underline{Does} Count?}

Good grades aren't enough.  The following are crucial:

\begin{itemstep}

\item Co-op/internship experience.  Very little chance of even getting
an interview for a technical job without this!

\item ``Brazil''  --- really \underline{knowing} the material, being
able to discuss it intelligently without review.

\item Good verbal abilities (NOT from taking more English classes).

\end{itemstep}

\end{slide}

}

\begin{slide}{UC vs. CSU}

\begin{itemize}

\item CSU has small class sizes (at UCD, a typical upper-division class
is 50 or 60, maybe more).

\item At a UC school, you are hearing it ``from the source.''  Most
faculty are doing cutting-edge research.  They lecture from their own
notes, not lecturing out of a textbook.

\end{itemize}

\end{slide}


\begin{slide}{UCD CS Faculty Research}

``Claims to fame'' (large, nationally-known research groups):

\begin{itemize}

\item Cryptography/Security

\item Graphics

\item Networks 

\item many nationally-known faculty in other fields

\end{itemize}

As a result, we have a lot of special courses other schools don't have
--- a 2-quarter UG sequence in networks, an UG security course, a host
of UG graphics courses, etc.

\end{slide}


\begin{slide}{Commitment to Students}

\begin{itemize}

\item 4 CS faculty have won the campus Distinguished Teaching Award.

\item 2 CS faculty have won the campus Best Faculty Adviser Award.

\end{itemize}

\end{slide}

\overlays{4}{

\begin{slide}{Guess What}

\bigskip
This isn't PowerPoint... 
\FromSlide{2}%
We're UNIX users. :-)

\bigskip
\FromSlide{3}%
UCD, like most major universities, is mainly UNIX-oriented.

\bigskip
\FromSlide{4}%
Most computer-related UC grads get jobs in full or partial UNIX shops.

\end{slide}

}

\begin{slide}{Linux}

Students are strongly encouraged to install Linux on their home PCs:

\begin{itemize}

\item Have the same software development environment at school, home.

\item Learn lots of valuable system admin skills!

\end{itemize}

\end{slide}

\overlays{2}{

\begin{slide}{Location, Location, Location}

\begin{itemstep}

\item ``1 hr. to Bay Area, 2 hrs. to Lake Tahoe...UCD is centrally
isolated.'' :-)

\item Proximity to Silicon Valley means:

Close relations between faculty and industry.

Most employers give high preference to locals, since they can
drive to interviews.

\end{itemstep}

\end{slide}

}
\end{document}
