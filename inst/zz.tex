\documentclass[pascal,pdf,colorBG,slideColor]{prosper}
%\hypersetup{pdfpagemode=FullScreen}
\usepackage{/usr/lib/R/share/texmf/Sweave}


\title{My Big Presentation}
\subtitle{which rocks}
\author{Jane Q. Doe}
\email{jdoe@colorado.edu}
\institution{Applied Mathematics\\University of Colorado}

\begin{document}
\maketitle


% \maketitle
\begin{slide}{Example}
A simple example that will run in any S engine: The integers from 1 to
10 are
\begin{Schunk}
\begin{Soutput}
 [1]  1  2  3  4  5  6  7  8  9 10
\end{Soutput}
\end{Schunk}

We can also emulate a simple calculator:
\begin{Schunk}
\begin{Sinput}
> 1 + 1
\end{Sinput}
\begin{Soutput}
[1] 2
\end{Soutput}
\begin{Sinput}
> 1 + pi
\end{Sinput}
\begin{Soutput}
[1] 4.141593
\end{Soutput}
\begin{Sinput}
> sin(pi/2)
\end{Sinput}
\begin{Soutput}
[1] 1
\end{Soutput}
\end{Schunk}
\end{slide}

\begin{slide}{First Slide Caption}
	This is the stuff on slide 1.
\end{slide}


\begin{slide}{Caption no.2}
	This is the stuff on slide 2.
	Note that $C=2\pi r$.
\end{slide}


\begin{slide}{Sphere equations}
	This is the stuff on slide 3.
	\begin{eqnarray}
		C & = & 2 \pi r \\
		A & = & 4 \pi r^2 \\
		V & = & \frac{4 \pi r^3}3
	\end{eqnarray}
\end{slide}

\end{document}
